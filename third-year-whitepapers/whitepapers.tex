\documentclass[12pt,a4paper]{article} 

%% -- Packages to use -- %%
	\usepackage[utf8]{inputnc}
	\usepackage{authblk}
	\usepackage{graphicx}
	\usepackage[toc,page]{appendix}
	\usepackage{hyperref}
	\usepackage[backend=bibtex, style=authoryear]{biblatex}
	\makeatletter
	\def\blx@maxline{77}
	\makeatother
	\usepackage[nottoc,numbib]{tocbibind}
%% -- Define path to .bib file -- %%
	\bibliography{example}
	\usepackage{float}
	\usepackage{listings}
	\usepackage{enumitem}

%% -- Generate the glossary -- %%
	\usepackage[acronym,nomain,nonumberlist]{glossaries}
	\makeglossaries

%% -- Acronym definitions {label}{abbreviation}{full} -- %%
	\newacronym{vm}{VM}{Virtual Machine}
	\newacronym{tls}{TLS}{Transport Layer Security}
	\newacronym{2fa}{2FA}{Two Factor Authentication}
	\newacronym{http}{HTTP}{Hypertext Transport Protocol}
	\newacronym{https}{HTTPS}{Hypertext Transport Protocol Secure}
	\newacronym{api}{API}{Application Programming Interface}
	\newacronym{ms}{MS}{Microsoft}
	\newacronym{sql}{SQL}{Structured Query Language}
	\newacronym{git}{Git}{Free and open source distributed version control system}
	\newacronym{aes}{AES}{ Advanced Encryption Standard}
	\newacronym{hsts}{HSTS}{HTTP Strict Transport Security}
	\newacronym{dip}{DLP}{Data Loss Prevention}
	\newacronym{ui}{UI}{User Interface}
	\newacronym{paas}{PaaS}{Platform as a Service}
	\newacronym{aws}{AWS}{Amazon Web Services}
	\newacronym{sqli}{SQLi}{Structured Query Language Injection}
	\newacronym{s3}{S3}{Amazon Simple Storage Service}
	\newacronym{rce}{RCE}{Remote Code Execution}
	\newacronym{url}{URL}{Uniform Resource Locator}
	\newacronym{cve}{CVE}{Common Vulnerabilities \& Exposures}
	\newacronym{ca}{CA}{Certificate Authority}
	\newacronym{owasp}{OWASP}{Open Web Application Security Project}
	
	
%% -- Set Title and Author Details -- %%
	\title{Hackers Handbook}
	\author{M. Hayden}
	\affil{BSc Ethical Hacking\\
		Abertay University\\
		Dundee, United Kingdom\\
		StudentNumber@abertay.ac.uk}			
	\date{December 4, 1985}

%% -- Beginning the main document -- %%
\begin{document}

%% -- For the title, abstract and table of contents use single column -- %%
	\pagenumbering{gobble}

%% -- Add the uni logo -- %%
	\begin{figure}
		\includegraphics[width=\linewidth]{img/unilogo}
	\end{figure}

%% -- Insert the title here -- %%
	\maketitle

% -- Abstract -- %
	\newpage				
		\begin{abstract}
		 A Bill To Make provision about the interception of communications, equipmentinterference and the acquisition and retention of communications data, bulk personal datasets and other information; to make provision about the treatment of material held as a result of such interception, equipment interference or acquisition or retention; to establish the Investigatory Powers Commissioner and other Judicial Commissioners and make provision about them and other oversight arrangements; to make further provision about investigatory powers and national security; to amend sections 3 and 5 of the Intelligence Services Act 1994; and for connected purposes.			
			
%% -- Leave a blank line to start a new paragraph -- %%					
		\end{abstract}

%% -- Table Of Contents -- %%
	\newpage
	\tableofcontents
	\newpage
				
%% -- Start Page Numbers -- %%
%% -- arabic = 1,2,3 -- %%	
		
	\pagenumbering{arabic}

%% -- Introduction -- %%
	\section{Introduction}
		 Some text used to fill space taken from the Investigatory Powers Bill (Parliament, 2016).
		
		More information about LaTeX \& Tex can be found on the Hack Soc wiki \footnote{https://hacksoc.co.uk/latex}.
			
%% -- Layout -- %%		
		\subsection{Layout}
%% -- Colin White Paper CMP319 2016/17 -- %%
			\subsubsection{Colin White Paper CMP319 2016/17}
				As a suggestion, your report should include:
				\begin{enumerate}
					\item{Introduction}
						\begin{enumerate}
							\item{Introduction to the report}
							\item{Aim of your work}
							\item{An overview of your methodology (i.e. each step that you have conducted to penetrate the network and the tools that you have used).}
						\end{enumerate}
					\item{Procedure and Results}
						\begin{enumerate}
							\item{This section should succinctly describe your practical work and findings}
							\item{Any results should be presented in an easy to read format.}
							\item{Include any relevant screenshots. These should be clearly labelled and referenced within the text of your report.}
						\end{enumerate}
					\item{References}
						\begin{enumerate}
							\item{References should be cited according the university's referencing criteria (http://www.abertay.ac.uk/media/Referencing.pdf)} 
							\item{You should cite appropriate references throughout your report.}
						\end{enumerate}
					\item{Appendices}
						\begin{enumerate}
							\item{Any large volume of information should be included in Appendices.}
						\end{enumerate}
				\end{enumerate}

%% -- David White Paper One CMP314 2016/17 -- %% 
			\subsubsection{David White Paper One CMP314 2016/17}
				The following structure is suggested:
				\begin{enumerate}
					\item{Introduction [15\%]}
						\begin{enumerate}
							\item{Abstract}
							\item{Introduction}
							\item{Overview of chosen area}
							\item{Your Objectives}
						\end{enumerate}
					\item{Procedure and Results [20\%]}
						\begin{enumerate}
							\item{An explanation of what you did and what you found}
							\item{A discussion of the practical steps included in your investigation}
							\item{The results should be easily followed and understood and should, where appropriate, include any relevant screenshots}
						\end{enumerate}
					\item{Discussion [20\%]}
						\begin{enumerate}
							\item{Critical evaluation of the results} 
							\item{Description of any further investigative work that could be performed in future research}
							\item{Any countermeasures.}
						\end{enumerate}
					\item{Conclusions [10\%]}
					\item {References \& Bibliography [10\%]}
				\end{enumerate}	
				
	\newpage

%% -- Background -- %%
		\subsection{Background}

%% -- Overview -- %%			
		\subsection{Overview}
		
%% -- LaTeX -- %%
			\subsubsection{LaTeX}
			"LaTeX, which is pronounced 'Lah-tech' or 'Lay-tech', is a document preparation system for high-quality typesetting. It is most often used for medium-to-large technical or scientific documents but it can be used for almost any form of publishing."\footnote{https://www.latex-project.org/about/}
			
%% -- Technology Two -- %%
			\subsubsection{Technology Two}
						
	\newpage	

%% -- Aims & Objectives  -- %%

	\section{Aims \& Objectives}
				
		\subsection{Aims}
			\begin{itemize}
				\item{Enslave Humanity}
				\item{Defeat C}
			\end{itemize}
				
		\subsection{Objectives}
			\begin{itemize}
				\item{Build AI}
					\begin{itemize}
						\item{Write AI framework}
						\item{Train it on cat pics}
						\item{Deploy TLS}
					\end{itemize}
					\item{Setup Windows 10 Virtual Machine in VMWare}
					\item{Setup Linux on Windows}
				\end{itemize}

	\clearpage
				
%% -- Procedure  -- %%
	\section{Procedure}
	
	\clearpage
			
%% -- Discussion & Comparison -- %% 
	\section{Discussion \& Comparison}									
									
%% -- Cats Are Awesome --  %%
		\subsection{Cats Are Awesome}
																					
	\clearpage

%% -- Conclusion -- %%											
	\section{Conclusion}
	
	\clearpage
	
%% -- Bliography -- %%
	\addcontentsline{toc}{section}{References}
	\printbibliography
	\nocite{*}
	
	\clearpage

%% -- Glossarie -- %% 
	\glsaddall
	\printglossaries
	\addcontentsline{toc}{section}{Glossaire}
	
	\clearpage
	
%% -- Appendix -- %%
	\begin{appendices}
	
		\section{Definition of "interception" etc.}
		Interception in relation to telecommunication systems
	
		\begin{enumerate}
			\item{For the purposes of this Act, a person intercepts a communication in the course of its transmission by means of a telecommunication system if, and only if - }
				\begin{enumerate}
					\item{the person does a relevant act in relation to the system, and}
					\item{the effect of the relevant act is to make any content of the communication available, at a relevant time, to a person who is not the sender or intended recipient of the communication.}
				\end{enumerate}
			For the meaning of "content" in relation to a communication, see section 233(6).
			\item{In this section ?relevant act?, in relation to a telecommunication system, means -}
				
				\begin{enumerate}
					\item{modifying, or interfering with, the system or its operation;}
					\item{monitoring transmissions made by means of the system;}
					\item{monitoring transmissions made by wireless telegraphy to or from apparatus that is part of the system.}
				\end{enumerate}

			\item{For the purposes of this section references to modifying a telecommunication system include references to attaching any apparatus to, or otherwise modifying or interfering with -}
				
				\begin{enumerate}
					\item{any part of the system, or}
					\item{any wireless telegraphy apparatus used for making transmissions to or from apparatus that is part of the system.}
				\end{enumerate}
				
			\item{In this section ?relevant time?, in relation to a communication transmitted by means of a telecommunication system, means -}
				
				\begin{enumerate}
					\item{any time while the communication is being transmitted, and}
					\item{any time when the communication is stored in or by the system (whether before or after its transmission).}
				\end{enumerate}		
		\end{enumerate}
	
	\end{appendices}	
	
%% -- END OF DOCUMENT -- %% 
\end{document}

											%% - 0 - %%
